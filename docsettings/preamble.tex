\ifsetCustomMargin
  \RequirePackage[left=37mm,right=30mm,top=35mm,bottom=30mm]{geometry}
  \setFancyHdr % To apply fancy header after geometry package is loaded
\fi

% Add spaces between paragraphs
%\setlength{\parskip}{0.5em}
% Ragged bottom avoids extra whitespaces between paragraphs
\raggedbottom
% To remove the excess top spacing for enumeration, list and description
%\usepackage{enumitem}
%\setlist[enumerate,itemize,description]{topsep=0em}

% ******************* Fonts (like different typewriter fonts etc.)*************
% Add `customfont' in the document class option to use this section

\ifsetCustomFont
  % Set your custom font here and use `customfont' in options. Leave empty to
  % load computer modern font (default LaTeX font).
  %\RequirePackage{helvet}

  % For use with XeLaTeX
  %  \setmainfont[
  %    Path              = ./libertine/opentype/,
  %    Extension         = .otf,
  %    UprightFont = LinLibertine_R,
  %    BoldFont = LinLibertine_RZ, % Linux Libertine O Regular Semibold
  %    ItalicFont = LinLibertine_RI,
  %    BoldItalicFont = LinLibertine_RZI, % Linux Libertine O Regular Semibold Italic
  %  ]
  %  {libertine}
  %  % load font from system font
  %  \newfontfamily\libertinesystemfont{Linux Libertine O}
\fi

% **************************** Custom Packages ********************************
% ************************* Algorithms and Pseudocode **************************
%\usepackage{algpseudocode}

% ********************Captions and Hyperreferencing / URL **********************
% Captions: This makes captions of figures use a boldfaced small font.
%\RequirePackage[small,bf]{caption}

\RequirePackage[labelsep=space,tableposition=top]{caption}
\renewcommand{\figurename}{Fig.} %to support older versions of captions.sty

% *************************** Graphics and figures *****************************

\usepackage{rotating}
\usepackage{wrapfig}
\usepackage{float}
\restylefloat{figure}
\usepackage{subcaption}
\usepackage{booktabs}
\usepackage{multirow}
\usepackage{multicol}
\usepackage{longtable}
\usepackage{tabularx}
\usepackage{siunitx} % use this package module for SI units

% ******************************* Line Spacing *********************************
% \doublespacing
% \onehalfspacing
% \singlespacing

% ************************ Formatting / Footnote *******************************
% Don't break enumeration (etc.) across pages in an ugly manner (default 10000)
%\clubpenalty=500
%\widowpenalty=500
%\usepackage[perpage]{footmisc} %Range of footnote options

% *************************** Bibliography  and References ********************
%\usepackage{cleveref} %Referencing without need to explicitly state fig /table

% Add `custombib' in the document class option to use this section
\ifuseCustomBib
   \RequirePackage[square, sort, numbers, authoryear]{natbib} % CustomBib

% If you would like to use biblatex for your reference management, as opposed to
% the default `natbibpackage` pass the option `custombib` in the document class.
% Comment out the previous line to make sure you don't load the natbib package.
% Uncomment the following lines and specify the location of references.bib file

%\RequirePackage[backend=biber, style=numeric-comp, citestyle=numeric, sorting=nty, natbib=true]{biblatex}
%\bibliography{References/references} %Location of references.bib only for biblatex

\fi

% changes the default name `Bibliography` -> `References'
\renewcommand{\bibname}{./myrefs}

% ******************************** Roman Pages *********************************
\newenvironment{romanpages}{
  \setcounter{page}{1}
  \renewcommand{\thepage}{\roman{page}}}
{\newpage\renewcommand{\thepage}{\arabic{page}}}

% ************************* User Defined Commands ******************************
% Common definitions of the work.
\newcommand{\autor}{AUTHOR's NAME}
\newcommand{\titulo}{TITLE OF PROJECT}
\newcommand{\subtitulo}{Some subtitle?}

% For macros you define inside your own document
\newcommand{\matlab}{\textsc{Matlab}\textsuperscript{\textregistered}}
\newcommand{\ie}{\emph{i.e.}}

% For math functions you make yourself
\DeclareMathOperator{\mode}{mode}
\DeclareMathOperator{\mean}{mean}
\DeclareMathOperator{\std}{std}

% For definitions and theorems
\newtheorem{teo}{Theorem}
\newtheorem{dfn}{Definition}
% ---------------------------------

% *********** To change the name of Table of Contents / LOF and LOT ************
%\renewcommand{\contentsname}{My Table of Contents}
%\renewcommand{\listfigurename}{My List of Figures}
%\renewcommand{\listtablename}{My List of Tables}

% ********************** TOC depth and numbering depth *************************
\setcounter{secnumdepth}{2}
\setcounter{tocdepth}{2}

% ******************************* Nomenclature *********************************
%\renewcommand{\nomname}{Symbols}

% ********************************* Appendix ***********************************
%\renewcommand{\appendixtocname}{List of appendices}
%\renewcommand{\appendixname}{Appndx}

% *********************** Configure Draft Mode **********************************
% Uncomment to disable figures in `draftmode'
%\setkeys{Gin}{draft=true}  % set draft to false to enable figures in `draft'

% These options are active only during the draft mode
% Default text is "Draft"
%\SetDraftText{DRAFT}

% Default Watermark location is top. Location (top/bottom)
%\SetDraftWMPosition{bottom}

% Draft Version - default is v1.0
%\SetDraftVersion{v1.1}

% Draft Text grayscale value (should be between 0-black and 1-white)
% Default value is 0.75
%\SetDraftGrayScale{0.8}

% ******************************** Todo Notes **********************************
\ifsetDraft
	\usepackage[colorinlistoftodos]{todonotes}
	\newcommand{\mynote}[1]{\todo[author=kks32,size=\small,inline,color=green!40]{#1}}
\else
	\newcommand{\mynote}[1]{}
	\newcommand{\listoftodos}{}
\fi

% Example todo: \mynote{Hey! I have a note}
